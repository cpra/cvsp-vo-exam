
\documentclass[paper=A4,11pt]{scrartcl}

\usepackage[utf8]{inputenc}
\usepackage[T1]{fontenc}

\usepackage{fixltx2e}
\usepackage{lmodern}
\usepackage{microtype}

\usepackage{xspace}
\usepackage[english]{babel}
\usepackage{csquotes}

\usepackage{amsmath}
\usepackage{amssymb}

\usepackage{graphicx}
\usepackage[margin=1em,labelfont=bf]{caption}
\usepackage{booktabs}
\usepackage{subcaption}

\usepackage[backend=biber]{biblatex}
\usepackage[bookmarks=false,colorlinks=false,pdfborder={0 0 0},linkbordercolor={0 0 0}]{hyperref}

% make title fonts serif
\setkomafont{sectioning}{\rmfamily\bfseries}

% suppress In: <journal>
\renewbibmacro{in:}{}

% load bibtex files
\addbibresource{report.bib}

% bold vectors
\renewcommand{\vec}[1]{\ensuremath{\mathbf{#1}}}

% vector shortcuts
\newcommand{\vx}{\vec{x}}
\newcommand{\vw}{\vec{w}}
\newcommand{\vy}{\vec{y}}
\newcommand{\vz}{\vec{z}}

\DeclareMathOperator*{\pr}{Pr}
\DeclareMathOperator*{\argmax}{arg\,max}

\begin{document}

\title{\Large 183.586 Computer Vision Systems Programming VO}
\subtitle{Exam Information}
\author{Christopher Pramerdorfer\\\small Computer Vision Lab, Vienna University of Technology}
\maketitle

\section{List of Questions} % (fold)
\label{sec:list_of_questions}

1. What factors should be considered when choosing a programming language?

\bigskip\noindent
2. Is there a \enquote{best} programming language? Why (not)?

\bigskip\noindent
3. What is the relation between image processing and Computer Vision (CV)?

\bigskip\noindent
4. What is the purpose of image processing? Name some examples.

\bigskip\noindent
5. Why should one not think in terms of algorithms when approaching CV problems?

\bigskip\noindent
6. What is the preferred way to approach CV problems and why?

\bigskip\noindent
7. What are the three steps to model-based CV solutions?

\bigskip\noindent
8. What is the difference between a model and an algorithm?

\bigskip\noindent
9. What is numerical optimization and how do iterative methods work?

\bigskip\noindent
10. How are images formed? Describe the pinhole camera model.

\bigskip\noindent
11. How does stereo reconstruction work?

\bigskip\noindent
12. What are depth sensors and how do they work?

\bigskip\noindent
13. Briefly describe some applications that utilize scene geometry.

\bigskip\noindent
15. What are the three discussed steps of 3D reconstruction?

\bigskip\noindent
16. Why is depth data well-suited for person detection?

\bigskip\noindent
17. How does Kinect's player pose estimation work?

\bigskip\noindent
18. What is a random forest and how does it work?

\bigskip\noindent
19. What kinds of object recognition are there and why is it challenging?

\bigskip\noindent
20. How does instance recognition of rigid objects work?

\bigskip\noindent
21. What are local features and why are they suitable for instance recognition?

\bigskip\noindent
22. How does Viola \& Jones' face detector work?

\bigskip\noindent
23. How does the bag of words model work and what is it used for?

\bigskip\noindent
24. What is the limitation of \enquote{traditional} object recognition methods?

\bigskip\noindent
25. What is deep learning and how are such models structured?

\bigskip\noindent
26. What is a convolutional neural network? How does it differ from a traditional MLP?

\bigskip\noindent
27. What is a convolutional layer?

\bigskip\noindent
28. How are convolutional neural networks structured?

\bigskip\noindent

\printbibliography

\end{document}
