
\documentclass[paper=A4,11pt]{scrartcl}

\usepackage[utf8]{inputenc}
\usepackage[T1]{fontenc}

\usepackage{fixltx2e}
\usepackage{lmodern}
\usepackage{microtype}

\usepackage{xspace}
\usepackage[english]{babel}
\usepackage{csquotes}

\usepackage{amsmath}
\usepackage{amssymb}

\usepackage{graphicx}
\usepackage[margin=1em,labelfont=bf]{caption}
\usepackage{booktabs}
\usepackage{subcaption}

\usepackage[backend=biber]{biblatex}
\usepackage[bookmarks=false,colorlinks=false,pdfborder={0 0 0},linkbordercolor={0 0 0}]{hyperref}

% make title fonts serif
\setkomafont{sectioning}{\rmfamily\bfseries}

% suppress In: <journal>
\renewbibmacro{in:}{}

% load bibtex files
\addbibresource{report.bib}

% bold vectors
\renewcommand{\vec}[1]{\ensuremath{\mathbf{#1}}}

% vector shortcuts
\newcommand{\vx}{\vec{x}}
\newcommand{\vw}{\vec{w}}
\newcommand{\vy}{\vec{y}}
\newcommand{\vz}{\vec{z}}

\DeclareMathOperator*{\pr}{Pr}
\DeclareMathOperator*{\argmax}{arg\,max}

\begin{document}

\title{\Large 183.586 Computer Vision Systems Programming VO}
\subtitle{Exam Information}
\author{Christopher Pramerdorfer\\\small Computer Vision Lab, Vienna University of Technology}
\maketitle

\section{Exam Information} % (fold)
\label{sec:exam_information}

The exam is oral, takes around 15 minutes, and can be taken in German or English. I will ask you four random questions from the list below. You should be able to answer each question in your own words to show that you understand the discussed topics, but you don't have to know formulas or source code. There are five categories of questions (see below) and each question I ask will be from a different category.\footnote{The script I will use for selecting questions is available at \url{https://github.com/cpra/cvsp-vo-exam}.}

For preparation, I suggest that you take a look at the comments to the lecture slides, which are available at \url{https://github.com/cpra/cvsp-vo-slides} (the \texttt{.tex} files). If you don't understand something based on the slides and comments, consult the references given on the corresponding lecture slides. Good luck!

\pagebreak

\section{List of Questions} % (fold)
\label{sec:list_of_questions}

The exam questions are as follows.

\subsubsection*{Introduction \& Image Processing} % (fold)
\label{ssub:introduction_image_processing}

\begin{itemize}
    \item What is Computer Vision (CV) and why is it important? Name some applications.
    \item What is the purpose of image processing? Briefly describe some image processing operations. What is the relation between image processing and CV?
\end{itemize}

\subsubsection*{Programming Languages \& Libraries} % (fold)

\begin{itemize}
    \item Describe how the programming languages we covered in the lecture differ, and what libraries there are for CV-related tasks.
    \item Which factors should be considered when choosing a programming language? Is there a best programming language? Why (not)?
\end{itemize}

\subsubsection*{Approaching CV Problems} % (fold)

\begin{itemize}
    \item Explain three things every CV problem consists of and the steps to solving CV problems. Describe and \enquote{solve} an example problem in this manner.
    \item What is a model and what does it do? What kinds of models are there? How is machine learning related to all this?
    \item We covered two approaches to selecting models. What are their differences and when are they applicable?
\end{itemize}

\subsubsection*{Specific Object Recognition} % (fold)

\begin{itemize}
    \item What is object recognition and why is it challenging? Explain the taxonomy used in the lecture using own examples.
\end{itemize}

\begin{itemize}
    \item How does detection of planar rigid objects work? Name some example applications. What are $\vx$ and $\vw$? Which model is applicable and why? Explain how detection can be performed in an automatic manner.
    \item What is the relation between corresponding points on a nonplanar object in two images? How can this relation be used to automatically find correspondences and for estimating 3D coordinates? What must be known for this to work?
    \item Explain the basic idea behind constellation models. Why are they useful and what are they used for? Describe an example application that favors such models, and explain how the model could look like in this application.
\end{itemize}

\subsubsection*{Object Category Recognition} % (fold)

\begin{itemize}
    \item What is the bag of words representation and how is it obtained? Explain the steps necessary to perform classification on this basis.
    \item Explain how face detection works. Which properties are desired? What are Haar features, what do they encode, and why can they be computed so efficiently? What is boosting, how does it work, and how can it be used for face detection?
    \item What is deep learning and what are the motivations for it? How do deep learning models look like? What are Convolutional Neural Networks (CNNs)? Which conditions must apply for CNNs to perform well?
    \item Describe the structure two-layer Perceptrons and their properties. Why are such models not applicable for deep learning? What are convolutional layers? How and why can they be used to realize deep models?
\end{itemize}

\pagebreak

\section{Fragenkatalog} % (fold)
\label{sec:fragenkatalog}

Die Prüfungsfragen lauten wie folgt.

\subsubsection*{Einleitung \& Image Processing} % (fold)

\begin{itemize}
    \item Was ist Computer Vision (CV) und warum ist es wichtig? Nennen Sie Beispielanwendungen.
    \item Was ist der Zweck von Image Processing? Beschreiben Sie kurz verschiedene Image Processing Operationen. Wie stehen Image Processing und CV in Zusammenhang?
\end{itemize}

\subsubsection*{Programmiersprachen \& Bibliotheken} % (fold)

\begin{itemize}
    \item Beschreiben Sie inwiefern sich die in der Vorlesung besprochenen Programmiersprachen voneinander unterscheiden und welche CV Bibliotheken existieren.
    \item Welche Faktoren beeinflussen die Wahl der Programmiersprache? Existiert eine beste Programmiersprache? Warum (nicht)?
\end{itemize}

\subsubsection*{Lösen von CV Problemen} % (fold)

\begin{itemize}
    \item Beschreiben Sie die drei Dinge, aus denen jedes CV Problem besteht, und die in der Vorlesung besprochenen Schritte zur Lösung solcher Probleme. Beschreiben und \enquote{lösen} Sie ein Beispielproblem auf diese Weise.
    \item Was ist ein Modell und welchen Zweck hat es? Welche Arten von Modellen gibt es? Wie steht Machine Learning in Relation dazu?
    \item In der Vorlesung wurden zwei Ansätze zur Wahl von Modellen besprochen. Wie unterscheiden sie sich und wann sind sie zutreffend?
\end{itemize}

\subsubsection*{Erkennung spezifischer Objekte} % (fold)

\begin{itemize}
    \item Was ist Objekterkennung und welche Herausforderungen gibt es? Beschreiben Sie die besprochenen Objekterkennungsarten anhand eigener Beispiele.
    \item Wie können planare und starre Objekte erkannt werden? Nennen Sie Beispielanwendungen. Was beschreiben $\vx$ und $\vw$? Welches Modell ist angemessen? Beschreiben Sie, wie solche Objekte automatisch detektiert werden können.
    \item Wie lautet die Relation zwischen korrespondierenden Punkten auf nicht-planaren Objekten in zwei Bildern? Wie kann diese Relation dazu benutzt werden, um solche Korrespondenzen automatisch zu finden und um 3D Koordinaten zu berechnen? Was muss dazu bekannt sein?
    \item Beschreiben Sie die grundlegende Idee hinter Constellation Models. Warum sind solche Modelle sinnvoll und wozu werden sie verwendet? Beschreiben Sie eine Beispielanwendung und erklären Sie, wie ein solches Modell aussehen könnte.
\end{itemize}

\subsubsection*{Erkennung von Objektkategorien} % (fold)

\begin{itemize}
    \item Was beschreibt die Bag of Words Repräsentation und wie wird sie berechnet? Erklären Sie die Schritte, die notwendig sind, um damit Bilder zu klassifizieren.
    \item Erklären Sie wie Gesichtserkennung funktioniert. Welche Eigenschaften sind gewünscht? Was sind Haar Features, was beschreiben sie und warum können Sie so effizient berechnet werden? Was ist Boosting, wie funktioniert es und wie kann es zur Gesichtserkennung verwendet werden?
    \item Was ist Deep Learning und was ist der Zweck? Wie sehen Deep Learning Modelle aus? Was sind Convolutional Neural Networks (CNNs)? Welche Bedingungen müssen erfüllt sein, damit CNNs gut funktionieren?
    \item Beschreiben Sie die Struktur eines Perceptrons mit zwei Schichten. Warum sind solche Modelle nicht für Deep Learning geeignet? Was sind Convolutional Layers? Wie und warum können sie verwendet werden um \enquote{tiefe} Modelle zu realisieren?
\end{itemize}

\end{document}
