
\documentclass[paper=A4,11pt]{scrartcl}

\usepackage[utf8]{inputenc}
\usepackage[T1]{fontenc}

\usepackage{fixltx2e}
\usepackage{lmodern}
\usepackage{microtype}

\usepackage{xspace}
\usepackage[english]{babel}
\usepackage{csquotes}

\usepackage{amsmath}
\usepackage{amssymb}

\usepackage{graphicx}
\usepackage[margin=1em,labelfont=bf]{caption}
\usepackage{booktabs}
\usepackage{subcaption}

\usepackage[backend=biber]{biblatex}
\usepackage[bookmarks=false,colorlinks=false,pdfborder={0 0 0},linkbordercolor={0 0 0}]{hyperref}

% make title fonts serif
\setkomafont{sectioning}{\rmfamily\bfseries}

% suppress In: <journal>
\renewbibmacro{in:}{}

% load bibtex files
\addbibresource{report.bib}

% bold vectors
\renewcommand{\vec}[1]{\ensuremath{\mathbf{#1}}}

% vector shortcuts
\newcommand{\vx}{\vec{x}}
\newcommand{\vw}{\vec{w}}
\newcommand{\vy}{\vec{y}}
\newcommand{\vz}{\vec{z}}

\DeclareMathOperator*{\pr}{Pr}
\DeclareMathOperator*{\argmax}{arg\,max}

\begin{document}

\title{\Large 183.586 Computer Vision Systems Programming VO}
\subtitle{Exam Information}
\author{Christopher Pramerdorfer\\\small Computer Vision Lab, Vienna University of Technology}
\maketitle

\section{Exam Information} % (fold)
\label{sec:exam_information}

The exam is oral, takes around 15 minutes, and can be taken in German or English. I will ask you four random questions from the list below. You should be able to answer each question in your own words to show that you understand the discussed topics, but you don't have to know formulas or source code. There are seven categories of questions (see below) and each question I ask you will be from a different category.\footnote{The script I use for selecting questions is available at \url{https://github.com/cpra/cvsp-vo-exam}.}

For preparation, I suggest that you take a look at the comments to the lecture slides, which are available at \url{https://github.com/cpra/cvsp-vo-slides} (the \texttt{.tex} files). If you don't understand something from the slides and comments, consult the references given on the corresponding lecture slides. Good luck!



\section{List of Questions} % (fold)
\label{sec:list_of_questions}

The exam questions are as follows.

\subsubsection*{Languages \& Libraries} % (fold)

1. What factors should be considered when choosing a programming language?

\bigskip\noindent
2. Is there a \enquote{best} programming language? Why (not)?


\subsubsection*{Image Processing} % (fold)

3. What is the relation between image processing and Computer Vision (CV)?

\bigskip\noindent
4. What is the purpose of image processing? Name some examples.


\subsubsection*{Models vs.\ Algorithms} % (fold)

5. Why should one not think in terms of algorithms when approaching CV problems?

\bigskip\noindent
6. What is the preferred way to approach CV problems and why?

\bigskip\noindent
7. What are the three steps to model-based CV solutions?

\bigskip\noindent
8. What is the difference between a model and an algorithm?

\bigskip\noindent
9. What is numerical optimization and how do iterative methods work?


\subsubsection*{3D Vision} % (fold)

10. How are images formed? Describe the pinhole camera model.

\bigskip\noindent
11. How does stereo reconstruction work?

\bigskip\noindent
12. What are depth sensors and how do they work?


\subsubsection*{3D Vision Applications} % (fold)

13. What are the three discussed steps of 3D reconstruction?

\bigskip\noindent
14. How does Kinect's player pose estimation work?

\bigskip\noindent
15. What is a random forest and how does it work?


\subsubsection*{Object Recognition 1} % (fold)

16. What kinds of object recognition are there and why is it challenging?

\bigskip\noindent
17. How does instance recognition of rigid objects work?

\bigskip\noindent
18. How does Viola \& Jones' face detector work?

\bigskip\noindent
19. How does the bag of words model work and what is it used for?


\subsubsection*{Object Recognition 2} % (fold)

20. What is the limitation of \enquote{traditional} object recognition methods?

\bigskip\noindent
21. What is deep learning and how are such models structured?

\bigskip\noindent
22. What is a convolutional neural network and how is it structured?



\section{Fragenkatalog} % (fold)
\label{sec:fragenkatalog}

Die Prüfungsfragen lauten wie folgt.

\subsubsection*{Languages \& Libraries} % (fold)

1. Welche Faktoren sollten bei der Wahl einer Programmiersprache beachtet werden?

\bigskip\noindent
2. Gibt es eine \enquote{beste} Programmiersprache? Warum (nicht)?


\subsubsection*{Image Processing} % (fold)

3. Wie stehen Image Processing und Computer Vision (CV) in Zusammenhang?

\bigskip\noindent
4. Was ist der Zweck von Image Processing? Nennen Sie Beispiele.


\subsubsection*{Models vs.\ Algorithms} % (fold)

\bigskip\noindent
5. Warum soll ein CV Lösungsansatz nicht auf Basis von Algorithmen gewählt werden?

\bigskip\noindent
6. Was ist der bevorzugte Lösungsansatz für CV Problem und warum?

\bigskip\noindent
7. Wie lauten die drei Schritte modellbasierter CV Lösungen?

\bigskip\noindent
8. Was ist der Unterschied zwischen einem Modell und einem Algorithmus?

\bigskip\noindent
9. Was ist numerische Optimierung und wie funktionieren iterative Methoden?


\subsubsection*{3D Vision} % (fold)

10. Wie entstehen Bilder? Beschreiben Sie das Lochkameramodell.

\bigskip\noindent
11. Wie funktioniert Stereo-Rekonstruktion?

\bigskip\noindent
12. Was sind Tiefensensoren und wie funktionieren sie?


\subsubsection*{3D Vision Applications} % (fold)

13. Wie lauten die drei besprochenen Schritte zur 3D Rekonstruktion?

\bigskip\noindent
14. Wie funktioniert die Haltungserkennung der Kinect?

\bigskip\noindent
15. Was ist ein Random Forest und wie funktioniert er?


\subsubsection*{Object Recognition 1} % (fold)

16. Welche Arten der Objekterkennung gibt es? Warum ist Objekterkennung schwierig?

\bigskip\noindent
17. Wie funktioniert die Instanzenerkennung von starren Objekten?

\bigskip\noindent
18. Wie funktioniert Viola \& Jones' Gesichtsdetektor?

\bigskip\noindent
19. Wie funktioniert das Bag of Words Modell und wozu wird es verwendet?


\subsubsection*{Object Recognition 2} % (fold)

20. Was ist die Einschränkung \enquote{traditioneller} Objekterkennungsmethoden?

\bigskip\noindent
21. Was ist Depp Learning und wie sehen solche Modelle aus?

\bigskip\noindent
22. Was ist ein Convolutional Neural Network und wie ist es aufgebaut?


\printbibliography

\end{document}
